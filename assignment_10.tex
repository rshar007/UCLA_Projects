% Created 2015-11-29 Sun 20:52
\documentclass[11pt]{article}
\usepackage[utf8]{inputenc}
\usepackage[T1]{fontenc}
\usepackage{fixltx2e}
\usepackage{graphicx}
\usepackage{grffile}
\usepackage{longtable}
\usepackage{wrapfig}
\usepackage{rotating}
\usepackage[normalem]{ulem}
\usepackage{amsmath}
\usepackage{textcomp}
\usepackage{amssymb}
\usepackage{capt-of}
\usepackage{hyperref}
\usepackage{graphicx}
\usepackage{longtable}
\usepackage{float}
\usepackage{multicol}
\author{Ryan Sharif}
\date{\today}
\title{}
\hypersetup{
 pdfauthor={Ryan Sharif},
 pdftitle={},
 pdfkeywords={},
 pdfsubject={},
 pdfcreator={Emacs 24.5.1 (Org mode 8.3.2)}, 
 pdflang={English}}
\begin{document}

\tableofcontents

\section{Revealing the Mysteries of the Maya Script}
\label{sec:orgheadline9}

\subsection{Introduction}
\label{sec:orgheadline1}
Ancient writing systems from the  Egyptians to the Mayan civilizations
used  eloborate and  enormous character  sets to  record phonological,
morphological, and  sentential information.Because of the  enormity of
the character corpus, researchers working  on the translation of these
corpora have not been able to completely translate the surviving Mayan
literature,     despit      over     two     hundred      years     of
effort\cite{macri_new_2009}. Recent  advances in machine  learning and
other  statiscally   based  classification  algorithms   have  allowed
researchers to build  databases that allow researchers  the ability to
use   machine    translation   and   search   to    move   translation
ahead\cite{hu_multimedia_2015}. This  paper examines the  news article
\cite{bourquenoud_revealing_2015} and  paper \cite{hu_multimedia_2015}
which looks at this process in depth.


\subsection{The Mayan script and problems that researchers face}
\label{sec:orgheadline4}
Researchers working on  the translation of ancient  Mayan writing face
several problems: (i)  there is a limited amount of  digitized data to
work   with,   (ii)  photographs   and   other   recorded  media   are
deteriorating, (iii) many of the glyphs that were once in the original
pieces have  become obscured,  and (iv)  glyphs artistic  style varies
depending on  the region and  time the glyphs were  rendered. Although
the first problem---available digitized data---may seem the easiest to
remedy, researchers face problems related to storing large RAW images,
sharing the data, and cataloging them for easy search and retrieval.

The problems related to deterioration, loss of data, and variations in
artistic style are difficult problems that may benefit from recent
advances in computer vision. Before looking at 

\subsubsection{Similarities to English orthography}
\label{sec:orgheadline2}

\subsubsection{Difficulty in deciphering the characters}
\label{sec:orgheadline3}
The fact  that there are still  people who speak a  descdended form of
Mayan language helps  researchers decipher the meaning  of Mayan text;
however, one of the greatest difficulties we face in translating these
texts is  figuring out when  a glyph  represents a single  phoneme, an
entire word, or an entire sentence.

\subsection{High quality representations of the Mayan script digitized}
\label{sec:orgheadline7}

\subsubsection{Digitizing the work and the catalogue}
\label{sec:orgheadline5}

\subsubsection{Machine Learning}
\label{sec:orgheadline6}

\subsection{Conclusion}
\label{sec:orgheadline8}


\bibliographystyle{plain}
\bibliography{mybib}
\end{document}
