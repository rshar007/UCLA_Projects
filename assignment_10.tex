% Created 2015-11-29 Sun 17:22
\documentclass[11pt]{article}
\usepackage[utf8]{inputenc}
\usepackage[T1]{fontenc}
\usepackage{fixltx2e}
\usepackage{graphicx}
\usepackage{longtable}
\usepackage{float}
\usepackage{wrapfig}
\usepackage{rotating}
\usepackage[normalem]{ulem}
\usepackage{amsmath}
\usepackage{textcomp}
\usepackage{marvosym}
\usepackage{wasysym}
\usepackage{amssymb}
\usepackage{hyperref}
\tolerance=1000
\author{Ryan Sharif}
\date{\today}
\title{assignment\_10}
\hypersetup{
  pdfkeywords={},
  pdfsubject={},
  pdfcreator={Emacs 24.5.1 (Org mode 8.2.10)}}
\begin{document}

\maketitle
\tableofcontents

\section{Revealing the Mysteries of the Maya Script}
\label{sec-1}

\subsection{Introduction}
\label{sec-1-1}
Ancient writing systems from the  Egyptians to the Mayan civilizations
used  eloborate and  enormous character  sets to  record phonological,
morphological, and  sentential information.Because of the  enormity of
the character corpus, researchers working  on the translation of these
corpora have not been able to completely translate the surviving Mayan
literature,     despit      over     two     hundred      years     of
effort\cite{macri_new_2009}. Recent  advances in machine  learning and
other  statiscally   based  classification  algorithms   have  allowed
researchers to build  databases that allow researchers  the ability to
use   machine    translation   and   search   to    move   translation
ahead\cite{hu_multimedia_2015}. This  paper examines the  news article
\cite{bourquenoud_revealing_2015} and  paper \cite{hu_multimedia_2015}
which looks at this process in depth.


\subsection{The Mayan script and problems that researchers face}
\label{sec-1-2}

\subsubsection{Similarities to English orthography}
\label{sec-1-2-1}

\subsubsection{Difficulty in deciphering the characters}
\label{sec-1-2-2}
The fact  that there are still  people who speak a  descdended form of
Mayan language helps  researchers decipher the meaning  of Mayan text;
however, one of the greatest difficulties we face in translating these
texts is  figuring out when  a glyph  represents a single  phoneme, an
entire word, or an entire sentence.

\subsection{High quality representations of the Mayan script digitized}
\label{sec-1-3}

\subsubsection{Digitizing the work and the catalogue}
\label{sec-1-3-1}

\subsubsection{Machine Learning}
\label{sec-1-3-2}

\subsection{Conclusion}
\label{sec-1-4}

\bibliographystyle{plain}
\bibliography{mybib}
% Emacs 24.5.1 (Org mode 8.2.10)
\end{document}
