% Created 2015-12-04 Fri 19:28
\documentclass[a4paper,twoside,twocolumn]{article}
\usepackage[utf8]{inputenc}
\usepackage[T1]{fontenc}
\usepackage{fixltx2e}
\usepackage{graphicx}
\usepackage{longtable}
\usepackage{float}
\usepackage{wrapfig}
\usepackage{rotating}
\usepackage[normalem]{ulem}
\usepackage{amsmath}
\usepackage{textcomp}
\usepackage{marvosym}
\usepackage{wasysym}
\usepackage{amssymb}
\usepackage{hyperref}
\tolerance=1000
\usepackage{multicol}
\usepackage{multicol}
\author{Ryan Sharif}
\date{\today}
\title{Revealing the Mysteries of the Maya Script: Review}
\hypersetup{
  pdfkeywords={},
  pdfsubject={},
  pdfcreator={Emacs 24.5.1 (Org mode 8.2.10)}}
\begin{document}

\maketitle

\section{Introduction}
\label{sec-1}
The Maya language as it was spoken  in the 16th Century in the Yucatan
Peninsula  has  been lost  for  centuries;  however, the  Maya  script
endures. How  do archaelogists make  sense of  a system that  has been
lost  for  centuries?  Ancient  writing  systems  used  eloborate  and
enormous  character sets  to record  phonological, morphological,  and
sentential information.   The task  seems impossible.  Because  of the
enormity  of   the  character  corpus,  researchers   working  on  the
translation  of  these  corpora  have  not  been  able  to  completely
translate  the surviving  Mayan literature,  despite over  two hundred
years  of  effort\cite{macri_new_2009}.  Recent  advances  in  machine
learning  and   other  statiscally   based  algorithms   have  allowed
researchers to build databases that allow researchers us to search and
use  machine translation  make progress  in dechiphering  this writing
system.\cite{hu_multimedia_2015} This paper examines the ACM Tech news
article     \cite{bourquenoud_revealing_2015}    and     IEEE    paper
\cite{hu_multimedia_2015} which looks at this process in depth.

\section{The Mayan script and problems that researchers face}
\label{sec-2}
Researchers working on  the translation of ancient  Mayan writing face
several problems: (i)  there is a limited amount of  digitized data to
work   with,   (ii)  photographs   and   other   recorded  media   are
deteriorating, (iii) many of the glyphs that were once in the original
pieces have  become obscured,  and (iv)  glyphs artistic  style varies
depending on  the region and  time the glyphs were  rendered. Although
the first problem---available digitized data---may seem the easiest to
remedy, researchers face problems related to storing large RAW images,
sharing the data,  and cataloging them for easy  search and retrieval.
Despite these difficulties, the IEEE paper points out that researchers
working on the writing system have translated approximately 80\% of the
writing that survives\textasciitilde{}\cite{kettunen_introduction_2008}.


Although the  first problem---available digitized data---may  seem the
easiest to remedy, researchers face  problems related to storing large
RAW images, sharing the data, and  cataloging them for easy search and
retrieval. The  problems related to  deterioration, loss of  data, and
variations in artistic  style are difficult problems  that may benefit
from recent advances in computer  vision. Thus, the IEEE paper focuses
on  the  algorithms  that  have helped  to  overcome  these  difficult
problems.

\section{High quality representations of the Mayan script digitized}
\label{sec-3}
You need to  use different tools for different problems,  and the IEEE
paper highlights  the different approaches  the team used.   One thing
researchers  would like  to do,  especially for  problems relating  to
deterioration, is  automatically retrieve the most  likely sequence of
glyphs given  the context of  where they  appear, i.e., given  that we
have seen a  sequence of glyphs, and given that  one may be unreadable
which one has the highest probability  of occuring here. The team used
a dynamic programming algorithm,  the Viterbi algorithm, whose results
suggest it  is well suited for  the task.  At its  core, the algorithm
calculates  the most  likely sequence  of events/characters  to appear
given a  training set of  data. Because more monumental  writings were
present in the training set,  the algorithm made better predictions on
similar  monumental data;  it  performed poorer  on  data recorded  on
paper-like media.

Many of the  glyphs in the Mayan script look  similar, which typically
presents a problem  for automatic categorization. Moreover,  how do we
deal with variations in style?  As  mentioned in the article, the team
considered  this problem  to be  separate from  contextual prediction.
There are  times when researchers  want a  model that best  predicts a
glyph given  no other  context, especially  when dealing  with damaged
glyphs.  In their experimental results, the researchers concluded that
a model, which  used a pivot based method for  classifying glyphs.  At
its core, this method aligns images so that they can be compared using
clusters of features,  e.g., both glyphs have a  particular leg symbol
in the same spot. The best model using this method correctly predicted
90\% of the experimental data set.

\section{Discussion}
\label{sec-4}
Machine  learning  and  dynamic programming  algorithms  have  yielded
solutions to many problems in science.  This article shows us that the
humanities  can benefit  from incorporating  recent advances  to solve
similarly difficult  problems. Extrapolating  the results  achieved in
the  IEEE paper,  we  can imagine  incorporating  these techniques  in
Ancient  Egyptian, Mesopotamian,  and  other  ancient writing  systems
sitting in  drawers in  museums around  the world,  invigorating these
dead writing systems once more.

\bibliographystyle{plain}
\bibliography{mybib}
% Emacs 24.5.1 (Org mode 8.2.10)
\end{document}
